%*******************************************************
% Abstract
%*******************************************************
%\renewcommand{\abstractname}{Abstract}
\pdfbookmark[1]{Abstract}{Abstract}
% \addcontentsline{toc}{chapter}{\tocEntry{Abstract}}
\begingroup
\let\clearpage\relax
\let\cleardoublepage\relax
\let\cleardoublepage\relax

\chapter*{Abstract}
 The increase in zero-day attacks on computer systems poses new challenges for IT security research.
 Intrusion detection systems (IDS) are designed to alert operators to any attacks in real time.
 Anomaly-based IDS is specifically targeted to detect zero-day attacks as well.
 In this thesis, an anomaly-based Host-based Intrusion Detection System (HIDS) based on Long-Short-Term-Memory neural networks (LSTMs) is developed.
 For this purpose, the sequence of system call data is investigated.
 In a further iteration, novel methods are presented with which the sequences are extended by additional system call parameters.
 This work investigates whether the LSTMs are successful in anomaly-based HIDS.\@
 Additionally, it will be investigated which parameters can be used to improve the results of the developed algorithm.
 It was shown in this work that the LSTM approach provides competitive results, but does so at a significant computational cost.
 The developed additional parameters could achieve a significant improvement of the results and could also find their use in connection with other algorithms.

\vfill

\begin{otherlanguage}{ngerman}
\pdfbookmark[1]{Zusammenfassung}{Zusammenfassung}
\chapter*{Zusammenfassung}
Die Zunahme von Zero-Day Angriffen auf Computer Systeme stellt die IT-Sicherheitsforschung vor neue Herausforderungen.
Intrusion Detection System (IDS) sollen die Betreibenden möglichst in Echtzeit auf jegliche Angriffe hinweisen.
Mit anomaliebasierten IDS wird speziell darauf abgezielt auch Zero-Day Angriffe zu erkennen.
In dieser Arbeit wird ein anomaliebasiertes Host-based Intrusion Detection System (HIDS) basierend auf Long-Short-Term-Memory neuronalen Netzwerken (LSTMs) entwickelt.
Dafür wird die Sequenz von System Call Daten untersucht.
In einer weiteren Iteration werden neuartige Verfahren präsentiert, mit welchen die Sequenzen durch weitere System Call Parameter erweitert werden.
Es soll im Rahmen dieser Arbeit untersucht werden, ob die LSTMs in der anomaliebasierten HIDS erfolgreich sind.
Zusätzlich soll untersucht werden, welche Parameter verwendet werden können um die Ergebnisse des entwickelten Algorithmus zu verbessern.
Es konnte im Rahmen dieser Arbeit gezeigt werden, dass der LSTM-Ansatz konkurrenzfähige Ergebnisse liefert, dies aber mit einem erheblichen Berechnungsaufwand.
Die entwickelten zusätzlichen Parameter konnten eine deutliche Verbesserung der Ergebnisse erzielen und könnten auch im Zusammenhang mit anderen Algorithmen ihren Einsatz finden.
\end{otherlanguage}

\endgroup

\vfill
