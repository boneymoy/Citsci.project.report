%*****************************************
\chapter{Roadmap}\label{ch:examples}
%*****************************************

\section{Procedure based on Hackley's level of participation}

Participants can access contribution to the project in a step wise manner. 
Idea to iterate through Hackley's level of participation to reach more possible users.
In an early phase a contributor is not obligated to have any prior knowledge about the subject.
While helping with diverse Projects in the entry level, the contributor might feel comfortable
with accessing more specialized tasks. In the following the different phases of contribution is discussed
using \glqq \textit{Hackley's Level of Participation} \ \grqq.

\subsection{1. Phase}  
\begin{itemize}
    \item Presentation of pictures
    \item Explanation for decision of algorithm
    
    $\rightarrow$ Is algorithm right with decision?
    \item Motivation for further investigation of algorithms
    \item Getting to know machine learning algorithms
\end{itemize}

In the first phase users are presented with images and a label based on the algorithms decision.
The users are also be able to see a


\subsection{2. Phase}

\begin{itemize}
    \item show different pictures/videos/data
    \item User decides which data fits the given label best
    \item User as loss function replacement/addition
\end{itemize}

\subsection{3. Phase}

\begin{itemize}
    \item user can upload designed models
    \item other users can test models and upload possible attacks
\end{itemize}

\subsection{4. Phase}

\begin{itemize}
    \item Ranking of robustness of different models for different pruposes
    \item collaborative analysis
    \item Forum for discussions 
\end{itemize}

\section{Different Approaches of Analysing Citizen Science Projects}
%*****************************************
%*****************************************
%*****************************************
%*****************************************
%*****************************************
