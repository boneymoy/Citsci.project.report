%*****************************************
\chapter{Folgerungen}\label{ch:folgerungen}
%*****************************************
Nachdem die Ergebnisse der Versuchsreihen präsentiert wurden sollen im Folgenden Einschätzungen der Ergebnisse erfolgen.
Neben dem Vergleich zu bestehenden Arbeiten soll dabei auch ein Bezug auf den Einsatz in der Praxis gemacht werden.
\section{LSTM Ansatz}

Die Ergebnisse beim Einsatz von \ac{LSTM} in der anomaliebasierten \ac{HIDS} zeigen zwei Dinge sehr deutlich.
Der aus der \ac{NLP} Ansatz kann erfolgreich in der \ac{HIDS} eingesetzt werden.
Das \ac{LSTM} kann ein Sprachmodell der System Calls erstellen und damit erfolgreich System Calls vorhersagen.
Dies konnten auch andere Arbeiten zeigen, doch wie zuvor in \autoref{sec:related_nlp} beschrieben erfolgten die Auswertungen bisher auf veralteten Datensätzen, mit praxisfernen Metriken und oft ohne detaillierte Beschreibung des Aufbaus.
Zusätzlich kann aufgrund dessen kein direkter Vergleich der Architekturen der \acp{LSTM} gezogen werden.

Zumindest ohne Parameter konnten die Ergebnisse der Experimente dabei für die n-gram und der Embidding Größe keinen klarern Trend abbilden.
So können bei einer n-gram Größe von $n=10$ und einer Embedding Größe von $e=4$ ähnlich gute Ergebnisse erzielt werden wie bei $n=2$ und $e=4$.
Allerdings ist der Berechnungsaufwand gerade bei großem $n$.
Speziell bei sehr großen Szenarien hat dies in dieser Arbeit zu Ressourcenengpässen geführt.
So konnten wie zuvor Beschrieben einige Konfigurationen nicht auf dem größten Szenario des \acp{LID-DS} berechnet werden.
Ein weiterer Nachteil der Anomalieerkennung durch ein \ac{LSTM} besteht darin, dass ein GPU für eine effiziente Berechnung nötig ist.
Die größte Schwierigkeit mit einem \ac{LSTM} ist dabei, dass Parallelisierung nur geringfügigen Einfluss auf die Berechnungszeit hat.
Grundproblem dabei ist die Rekurrenz des Netzes. 
Die Ausgabe des letzten Zeitschrittes muss bekannt sein um die Ausgabe des aktuellen Zeitschrittes berechnen zu können.
So kann daraus gefolgert werden, dass der reine Einsatz von \acp{LSTM} nur auf der Sequenz der System Call Namen keine Verbesserung im Vergleich mit zum Beispiel der \ac{STIDE} Implementierung von Grimmer et al.\cite{IDSTHREADGRIMMER2021} erreicht werden kann.
Das liegt wahrscheinlich daran, dass die Muster in den Sequenzen nicht komplex genug sind, dass die Vorteile der \acp{LSTM} zu trage kommen.

Wie dies unter der Verwendung von Extraparametern aussieht wird im Folgenden betrachtet.

\section{Einsatz von Extraparametern}

Wie in \autoref{sec:erg_LSTM_extra} beschrieben kann eine deutliche Verbesserung der \ac{DR} sowie  der \ac{FP}-Rate mit dem Einsatz verschiedener Extraparameter erzielt werden.
So sind unter $9$ von $10$ der besten Ergebnisse nach \ac{DR} und $7$ von $10$ nach \ac{FP}-Rate Konfigurationen mit mindestens einem Extraparameter.
Im Vergleich zu den Ergebnissen ohne Extraparametern scheint hier ein Trend zu größeren n-grammen zu gehen.
Es ist jeweils nur noch eine Konfiguration unter den besten zehn mit einer n-gram Größe von $2$.
Im Vergleich zu drei bzw.\ vier der besten zehn bei Konfigurationen ohne Verwendung von Extraparametern.\marginpar{jeweils auf höchste \ac{DR} und niedrigste \ac{FP}-Rate bezogen}
Dies spiegelt eventuell die zunehmende Komplexität der Eingaben wieder.  
