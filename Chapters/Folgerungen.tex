%*****************************************
\chapter{Folgerungen}\label{ch:folgerungen}
%*****************************************
Nachdem die Ergebnisse der Versuchsreihen vorgestellt wurden, folgt nun eine Bewertung der Ergebnisse.
Dafür werden in \autoref{sec:folgerungen_LSTM} zunächst nur die Ergebnisse des \ac{LSTM} Ansatz besprochen.
Im Anschluß werden in \autoref{sec:folgerungen_extra} die Folgerungen aus den Experimenten mit Extraparametern erörtert. 
Schließlich werden in \autoref{sec:folgerungen_vgl} die Ergebnisse des Vergleichs anderer Arbeiten mit demselben Datensatz besprochen

\section{\NoCaseChange{\ac{LSTM}} Ansatz}\label{sec:folgerungen_LSTM}
\sectionmark{LSTM Ansatz}

Die Ergebnisse beim Einsatz von \acp{LSTM} in anomaliebasierten \acp{HIDS} zeigen zwei Dinge sehr deutlich.
%Der aus der \ac{NLP} bekannte Ansatz kann erfolgreich in der \ac{HIDS} eingesetzt werden.
Erstens, das \ac{LSTM} kann ein Sprachmodell der System Calls erstellen und damit erfolgreich System Calls vorhersagen.
Dies haben auch andere Arbeiten versucht zu zeigen, doch wie zuvor in \autoref{sec:related_nlp} beschrieben, erfolgten die Auswertungen bisher auf veralteten Datensätzen mit praxisfernen Metriken und oft ohne detaillierte Beschreibung des Aufbaus.
Aufgrund dessen kann leider kein direkter Vergleich der Architekturen der \acp{LSTM} gezogen werden.
Die Ergebnisse der Experimente konnten für die N-Gram und Embidding Größe keinen klaren Trend abbilden.
Bei einer N-Gram Größe von $n=10$ und einer Embedding Größe von $e=4$ konnten ähnlich gute Ergebnisse erzielt werden wie bei $n=2$ und $e=4$.\par\medskip

Zweitens zeigt sich allerdings, dass der Berechnungsaufwand gerade bei großem $n$ sehr hoch ist.
Speziell bei sehr großen Szenarien hat dies in dieser Arbeit zu Ressourcenengpässen geführt.
Es konnten, wie zuvor beschrieben, einige Konfigurationen nicht auf dem größten Szenario des \acp{LID-DS} berechnet werden.\par\medskip
Ein weiterer Nachteil der Anomalieerkennung durch ein \ac{LSTM} besteht darin, dass ein GPU für eine effiziente Berechnung nötig ist.
Hinzu kommt, dass mit einem \ac{LSTM} eine Parallelisierung nur geringfügigen Einfluss auf die Berechnungszeit hat.
Grundproblem dabei ist die Rekurrenz des Netzes. 
Die Ausgabe des letzten Zeitschrittes muss bekannt sein, um die Ausgabe des aktuellen Zeitschrittes berechnen zu können.
Aus den vorliegenden Ergebnissen kann, im Vergleich zur \ac{STIDE} Implementierung von Grimmer et al.~\cite{IDSTHREADGRIMMER2021}, der Einsatz von \acp{LSTM} in der beschriebenen Architektur, ohne die Verwendung von zusätzlichen Parametern, keine Verbesserung erzielen.\par\medskip

Daraus lässt sich schließen, dass \acp{LSTM} ohne die Verwendung zusätzlicher Parameter aufgrund des mit der Berechnung verbundenen Mehraufwands keine nennenswerten Vorteile bringen konnte.
% Das liegt wahrscheinlich daran, dass die Muster in den Sequenzen nicht komplex genug sind, dass die Aufwendige Implementierung der \acp{LSTM} zu trage kommen.

Wie dies unter der Verwendung von Extraparametern aussieht wird im Folgenden betrachtet.

\section{Einsatz von Extraparametern}\label{sec:folgerungen_extra}

Wie in \autoref{sec:erg_LSTM_extra} beschrieben, kann eine deutliche Verbesserung der \ac{DR} sowie  der \ac{FP}-Rate mit dem Einsatz verschiedener Extraparameter erzielt werden.
So sind $9$ unter den besten $10$ Ergebnissen bezüglich \ac{DR} und $7$ bezüglich \ac{FP}-Rate Konfigurationen mit mindestens einem Extraparameter.\par\medskip

Im Vergleich zu den Ergebnissen ohne Extraparametern ist hier ein Trend zu größeren N-Grammen erkennbar.
Es ist nur noch insgesamt eine Konfiguration unter den besten zehn Ergebnissen bezüglich \ac{DR} und \ac{FP} mit einer N-Gram Größe von $2$.
Im Vergleich zu insgesamt sieben \marginpar{drei bezüglich \ac{DR}, vier bezüglich \ac{FP}} der besten zehn bei Konfigurationen ohne Verwendung von Extraparametern.
Woraus sich schließen lässt, dass bei komplexeren Eingaben größere N-Gramme bessere Ergebnisse liefern.\par\medskip

Auffällig sind auch die Verteilungen der N-Gram Größen.
Unter Einsatz der Extraparameter kommt bezüglich \ac{DR} und \acp{FP} einmal eine Konfiguration mit $n<6$ vor, während dies bei Konfigurationen ohne Extraparametern sieben mal der Fall ist.\par\medskip

Abschließend lässt sich sagen, dass $n=10$ und $e=4$ sowie $n=6$ und $e=8$ die besten Konfigurationen bezüglich \ac{DR} und \ac{FP} darstellen.
Zusätzlich konnte zwar keine eindeutig beste Konfiguration für die Kombination der zusätzlichen Parametern gefunden werden, aber fast alle Kombinationen konnten eine Verbesserung der Ergebnisse erzielen.

\section{Vergleich anderer Arbeiten}\label{sec:folgerungen_vgl}

Der Vergleich mit anderen \ac{LSTM}-basierenden Arbeiten ist, wie bereits erwähnt, aufgrund der uneinheitlichen Datensätze schwierig.
Dennoch kann ein Vergleich zu der Arbeit von Grimmer et al.~\cite{IDSTHREADGRIMMER2021} gezogen werden.
Hierfür wurden die besten Konfiguration sowie das von Grimmer et al.~\cite{IDSTHREADGRIMMER2021} eingeführte 5-Level-Modell herangezogen.
Im Direktvergleich der besten Konfiguration in \autoref{tab:LSTM_stide_erg} ist signifikanter Rückgang der \acp{FP} zu erkennen, in \autoref{tab:LSTM_lvl} ist dies nicht mehr erkennbar.
Das \ac{LSTM} kann mit einer Auswahl an Algorithmen auf allen Leveln bis auf Level $5$ konkurrieren.
Dabei verwenden diese Algorithmen keine Extraparameter.
Der entworfene anomaliebasierte \ac{HIDS} Ansatz unter Verwendung eines \acp{LSTM} kann somit mit den in der Forschung verwendeten Algorithmen konkurrieren.
Wie die erlangten Erkenntnisse in Zukunft weiter ihren Einsatz finden können, soll im folgenden Kapitel untersucht werden.
