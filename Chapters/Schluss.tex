\chapter{Zusammenfassung und Ausblick}

Informationssicherheit in der IT-Welt ist vielen Angriffen einen Schritt hinterher.
Denn die Verteidigungssysteme beruhen darauf Signaturen von Angriffen auf Computersystemen wiederzuerkennen.
Sind die  Angriffe nicht bekannt können sie auch nicht erkannt werden.
Gleiches gilt bei Abwandlungen von bekannten Angriffen.
Um den Angriffen nicht immer hinterher zu sein wird versucht mit anomaliebasierten Verteidigungssysteme auch unbekannte Angriffe zu erkennen.
Statt eine Liste von Angriffen ständig up to date zu halten, wird das Normalverhalten eines oder mehrerer Systeme betrachtet.
Führt ein Angriff zu einer Abweichung des Normalverhaltens wird dieser erkannt und es können weitere Schritte eingeleitet werden.

\section{Zusammenfassung}

\acp{IDS} kommen in vielen Bereichen der IT-Welt zum Einsatz.
Häufig werden \ac{NIDS} verwendet, welche Netzwerkdaten untersuchen und damit mehrere Hosts gleichzeitig überwachen.
Um auch Angriffe auf einzelnen Hosts zu erkennen, reichen diese meist nicht aus.
Mithilfe von \acp{HIDS} kann dies erreicht werden.
Diese Arbeit beschäftigt sich mit \acp{HIDS} welche Anomalien in Sytem Call Streams erkennen. 
Da im ersten Schritt die lediglich die Sequenz der System Calls betrachtet wird, bietet es sich an von etablierte Verfahren der Mustererkennung zu profitieren.
Ein Bereich der Mustererkennung welcher sehr viel Fortschritt erfahren hat ist die \ac{NLP}.
Es stellt sich also die Frage ob bestimmte Verfahren die in der \ac{NLP} Verwendung finden sich auch in der Domäne der \ac{HIDS} als erfolgreich erweisen.
\acp{LSTM} zeigten sich in der Sprachverarbeitung sowie auch in verschiedenen Ansätzen der Zeitreihendaten als effektiv.
Es konnte in diese Arbeit gezeigt werden, dass sich diese Fortschritte in der \ac{NLP} auf die anomaliebasierte \ac{HIDS} übertragen lässt.
Dafür wurde mit dem \ac{W2V}-Verfahren ein ebenfalls aus der \ac{NLP} stammendes Embedding Verfahren verwendet um eine kontexterhaltende Kodierung der System Call Namen zu erhalten.
Des Weiteren wurden neue Verfahren zur Kodierung zusätzliche Informationen der System Calls entwickelt.
Diese zielen darauf ab zum einen die Rückgabewerte bestimmter lesender und schreibender System Calls zu überwachen, sowie den zeitliche Abstand zwischen zwei System Calls innerhalb eines Threads darzustellen. 
Zusätzlich wurden die von Grimmer et al.~\cite{IDSTHREADGRIMMER2021} \textit{Thread aware} n-gramme für Algorithmen mit Kontextwissen erweitert, indem ein Wechsel des Threads als solcher gekennzeichnet wird.
Um zu überprüfen ob sich die beschriebenen Verfahren als erfolgreich erweisen wurden ausführliche Experimente auf dem \ac{LID-DS} durchgeführt.
Dabei wurden verschiedene Parameter für die Eingabedaten untersucht.
Eine klare Tendenz für bestimmte Parameterkombinationen waren dabei nicht zu erkennen.
Allerdings wurde gezeigt, dass die \acp{LSTM} speziell mit Hinzunahme der Extraparameter konkurrenzfähige Ergebnisse liefern kann.
Der Größte Nachteil des präsentierten \ac{HIDS} Verfahren besteht in der aufwendigen Berechnung der \acp{LSTM}.

\section{Ausblick}
Einen Alternativen \ac{NLP} Ansatz, welcher sich bereits bei den meisten Übersetzungs und Sprachgenerierungsprogrammen durchgesetzt hat sind die Transformer.
Diese ermöglichen eine Verarbeitung der System Calls mit Einbindung des Kontexts die zusätzlich parallel Berechnet werden kann.

So kann eventuell der Vorteil der derExtraparameter können deutliche Verbesserung der Ergebnisse bringen.
Nur unter Einsatz der Extraparameter kann der \ac{LSTM} Ansatz mit anderen Algorithmen mithalten.
Evtl können diese Extraparameter auch Verbesserung der Ergebnisse bei anderen Algorithmen erzielen.

Evtl helfen Transformer dabei die Berechnungszeit zu verringern und bei gleichzeitigem Erhalt von Kontextwissen.
Sie ermöglichen parallele Ausführung der Sequenz.
Also speziell auch längere Sequenzen möglich.
