\chapter*{Acronyms}\label{ch:introduction} %************************************************
\begin{acronym}
  \acro{IDS}{Intrusion Detection System}
  \acro{HIDS}{Host-based Intrusion Detection System}
  \acro{NIDS}{Network Intrusion Detection System}
  \acro{SIDS}{Signature-based Intrusion Detection Systems}
  \acro{AIDS}{Anomaly-based Intrusion Detection Systems}
  \acro{ADFA-LD}{ADFA Linux Dataset}
  \acro{LSTM}{Long-Short-Term-Memory}
  % \acrop{LSTMs}{Long-Short-Term-Memory}
  \acro{RNN}{Rekurrente neuronale Netze}
  \acro{BSI}{Bundesamt fuer Sicherheit in der Informationstechnik}
  \acro{API}{Application Programming Interface}
  \acro{CVE}{Common Vulnerabilities and Exposures}
  \acro{CWE}{Common Weakness Enumeration}
  \acro{CNN}{Convolutional Neural Nets}
  \acro{NLP}{Natural Language Processing}
  \acro{ML}{Maschinelles Lernen}
  \acro{FP}{False Positive}
  \acro{FN}{False Negative}
  \acro{TN}{True Negative}
  \acro{TP}{True Positive}
  \acro{TIDE}{Time-Delay Embedding}
  \acro{STIDE}{Sequence Time-Delay Embedding}
  \acro{RIPPER}{Repeated Incremental Pruning to Produce Error Reduction}
  \acro{HMM}{Hidden Markov Model}
  \acro{DR}{Detektionsrate}
  \acro{OHE}{One-Hot-Encoding}
  \acro{W2V}{Word2Vec}
  \acro{LIDDS}{Leipzig Intrusion Detection Dataset}
\end{acronym}
