%************************************************
\chapter{Einführung}\label{ch:introduction} %************************************************
%*****************************************
%*****************************************
%*****************************************
%*****************************************
%*****************************************
\section{Einleitung}
Notes:

\begin{itemize}
    \item solarwinds as introduction
    \item use advances of sequence detection from NLP
    \item NIDS vs. HIDS
    \item signature vs.\ anomaly based
    \item Forrest et al 1996 erstmals syscall traces
    \item low level interactions between program and kernel
    \item syscall traces dont stop execution contrary to debuggers
    \item tracing virtually every linux without modifying source code
    \item whole system behaviour visible in kernel
\end{itemize}

Angriffe auf Computersysteme werden frequenter. 

Neue Methoden werden ben\"otigt um komplexere und neuartige Angriffe zu erkennen.

Erkennung von Angriffen \ac{IDS}.

Die häufig verwendeten auf Signaturen basierenden Abwehrmechanismen reichen nicht aus um viele drohende Gefahren abzuwenden.
Dies liegt hauptsächlich daran, dass weder Abwandlungen von bekannten Angriffen, noch unbekannte Angriffe erkannt werden können.
Zusätzlich müssen die Signaturen für jeden Angriffsvektor einzeln hinzugefügt werden.
Ein wesentlicher Vorteil liefert hier die Angriffserkennung über Anomalien.
Im Gegensatz zu dem erwähnten signaturbasierten Ansatz, muss dabei nicht jeder Angriff der abgewehrt werden soll bekannt sein.
%Stattdessen wird versucht das Normalverhalten eines Systems zu ermitteln und jegliche Abweichung als Anomalie und damit als Angriff einzustufen.
Stattdessen wird versucht ein Modell des zu erwartenden Normalverhalten des Systems zu erstellen.
Mit dem erstellten Modell sollen dann, möglichst in Echtzeit, Abweichungen bzw. Anomalien des erwarteten Verhalten signalisiert werden.
Eine der größten Herausforderungen die beim Einsatz von Anomalieerkennungen auftreten, besteht darin die Anzahl der Fehlalarme auf einem geringen Niveau zu halten. 
So kann bei großen Datenmengen schon eine sehr geringe Fehlalarmrate  zu einer für einen Analysten nicht zu bewältigenden Aufgabe werden. \cite{ANOMALY_SURVEY}
In dieser Arbeit ein \ac{HIDS} entwickelt werden, dabei werden die Daten auf dem zu überwachenden \textit{Host} abgegriffen werden.
Im Gegensatz dazu stehen zum Beispiel die \ac{NIDS}, hier werden die Daten auf Netzwerkebene und damit auch von mehreren Hosts überwacht.
\ac{HIDS} bieten Vorteile durch ihre feingranularität und bieten die Möglichkeit auch interne Attacken erkennen zu können.
Im realen Einsatz von \ac{HIDS}  besteht eine Schwierigkeit darin, dass das \ac{IDS} Zugriff auf den Kernel des zu überwachenden Systems benötigt.
Diese und weitere Herausforderungen die sich mit der praktischen Umsetzung in einem potentiellen Betrieb entstehen können sollen allerdings in dieser Arbeit nicht behandelt werden, da lediglich die Algorithmen selbst untersucht werden sollen.

Doch neben der Frage wo die Daten abgegriffen werden, muss geklärt werden welche Daten genutzt werden sollen.
Welche Daten ermöglichen es das dem System zugrunde liegene Verhalten zu beschreiben.
Dabei sollten sie zusätzlich so abstrakt sein, dass sie in vielen Systemen ihren Einsatz finden.
Eine häufig verwendete Information für die Charakterisierung von Systemen bieten zum Beispiel System-Logs~\cite{HE}.

In dieser Arbeit werden System-Calls verwendet.
Sie bieten eine sehr abstrakte Betrachtung auf Linux Betriebssystemebene.
Die Grundidee besteht darin, dass Programme auf einer Festplatte meist erst Schaden anrichten, sobald sie ausgeführt werden.
Dabei führen sie betriebssystemspezifische System-Calls aus, welche über verschiedene Tools wie zum Beispiel Sysdig~\cite{SYSDIG} ausgelesen werden können.
Die Schwierigkeit im Vergleich zu dem Untersuchen der Logs besteht darin, die großen Datenmengen zu bewältigen, welche schon bei kleineren Anwendungen anfallen.
Generell konnten Probleme in der Verarbeitung von sehr großen Datenmengen unter anderem durch die Verwendung selbst lernender Algorithmen erfolgreich angegangen werden.
Es gilt also in großen Datenmengen Zusammenhänge zwischen potentiell auseinanderliegenden Abfolgen von System-Calls zu erlernen.
Erfolgreich haben sich dabei \ac{LSTM} Netzwerke gezeigt.
Sie haben den Vorteil auch Zusammenhänge mit größerer zeitlicher Verzögerung noch zu erkennen~\cite{HOCHREITER} und können in unterschiedlichsten Architekturen einen Nutzen bringen. %\cite{SMAGULOVA}.
Da die \ac{LSTM} Netzwerke auch in der Spracherkennung erfolgreich sind~\cite{NLP_LSTM}, ensteht die Frage in wie weit auch andere Erkenntnisse aus der Welt der \ac{NLP} sich in den Bereich der Analyse von System-Calls übertragen lassen.



In verschiedenen Arbeiten wurde bereits die Abfolge von System-Calls betrachtet, um darin Anomalien zu erkennen. 
Doch nur in wenigen Arbeiten werden auch die Parameter zur Anomalieerkennung verwendet.
Eine der ersten Arbeiten von Forrest et al.~\cite{FORREST} betrachtet lediglich die Sequenzen der System-Calls.
Grimmer et al.\ versuchen die Bildung der Sequenzen durch Betrachtung der Thread-ID anzupassen~\cite{LIDS}, dies bindet Zusatzparameter zur Verbesserung der Sequenzbildung ein, aber dennoch werden \glqq nur \grqq{} Sequenzen betrachtet.
Maggi et al.\ verwenden zusätzlich auch Parameter und verweisen in ihrer Arbeit~\cite{MAGGI} auf diverse Ansätze.
In dieser Arbeit soll ebenfalls versucht werden durch die Hinzunahme von Parametern, wie zum Beispiel den Rückgabewerten (sofern vorhanden), die Erkennungsquote bzw.\ die \ac{FPR} des \ac{IDS} zu verbessern.


\section{Zielsetzung}

In dieser Arbeit sollen zwei Forschungsfragen verfolgt werden.
\begin{itemize}
    \item Kann der Erfolg von LSTM-Netzwerken in verschiedenen Bereichen auf die Erkennung von Anomalien in der Cyber-Sicherheit übertragen werden?
    \item Kann die Zunahme von Parametern bei der Anomalieerkennung mittels System-Calls eine Verbesserung bringen?

        $\rightarrow$ Welche Parameter kommen in Frage?
\end{itemize}
%Des Weiteren wird, je nach Erfolg der ersteren Fragen noch optional folgendes untersucht:
%\begin{itemize}
%    \item Können aktuelle Verbesserungen des Lernverhaltens durch GAN auch hier Anwendung finden?
%    \begin{itemize}
%        \item MAD-GAN \cite{LI}
%        \item VAE-MAD-GAN \cite{NIU}
%    \end{itemize}
%\end{itemize}

Um diese Forschungsfragen angemessen behandeln zu können müssen zunächst Grundlagen aus verschiedenen Bereichen gelegt werden.
Zum einen werden unterschiedliche Herangehensweisen zur Überwachung von Systemen betrachtet und erläutert wieso es für diese Anwendung sinnvoll ist eine Host-Based Intrusion Detection zu wählen.
Speziell soll auch beschrieben werden, warum sich System-Calls zur Überwachung von Computersystemen eignen.
Des Weiteren müssen Grundlagen für die in dem verwendeten Algorithmus verwendeten Techniken gelegt werden.
Dazu gehören Hauptsächlich Grundlagen zu rekurrenten neuronalen Netzen (RNN) sowie die Erweiterungen der LSTM Netzwerke.

Ein großer Teil der Implementierungsarbeit jedoch wird die Vorverarbeitung der Daten darstellen.
Diese soll mit der genaueren Untersuchung der Zusammensetzung der Techniken für den Algorithmus in einem weiteren Kapitel dargestellt werden.
Nachdem die verwendete Software analysiert wurde, wird eine Auswertung auf dem LID-DS~\cite{LID-DS} Datensatz durchgeführt.
Dieser bietet den Vorteil, dass in einer reproduzierbaren Art System Calls aufgenommen wurden.
Des Weiteren werden zusätzlich die System Call Parameter, wie zum Beispiel die \textit{Thread ID} zur Verfügung gestellt.

Im letzten Teil der Arbeit soll dann eine Schlussfolgerung aus den zuvor gewonnenen Ergebnissen gezogen werden. 
Hauptsächlich sollen die gestellten Forschungsfragen untersucht werden.
Konnte mit einem hinzugezogenen Parameter ein Mehrwert erzielt werden?
Bieten sich LSTM-Netzwerke auch für die Anomalieerkennung im IT-Sicherheitsbereich an?
