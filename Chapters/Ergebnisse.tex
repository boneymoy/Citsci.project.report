
\chapter{Ergebnisse}
\label{ch:erg}
\section{Optimale Parameter}

        \paragraph{Architektur}
            versch architekturen:
            Single Small 50 neuronen eine schicht
            Single Big 250 neuronen eine schicht
            multi small 20 neuronen 3 schichten
            multi big 50 neuronrn 3 schichten
            deep erste 50 sonst 20 6 schichten

            singlesmall 43\% von Deep
            insgesamt am schnellsten single small
            wie zu erwarten,  deep am langsamsten

        \paragraph{Hyperparameter}<++>
            aktivierungs funktion
            -> dense layer with softmax or tanh
            batch size
            learning rate
            optimizer

        \paragraph{Ngram Größe}
            ngram größer -> langsamer

        \paragraph{Embedding}

            overhead berechnung embedding, muss allerdings nur einmal berechnet werden
            zu erkennen w2v mit embedding size = 2  und window = 4 wesentlich schneller
            embedding größer -> langsamer

            vergleich ngram
            im schnitt mit ngram gr 2 84\% von ngr 3 und 

            w2v bringt entscheidenden Vorteil gegenüber ohe:
            Jeweils vergleich der selben parameter außer w2v vs ohe:
            Single small w2v nur 30\% der zeit gegenüber single small ohe
            bei mulit w2v sogar nur 13\%
            im mittel über alle architekturen 21.5\% der Zeit von ohe bei verwendung w2v

        \paragraph{Architektur}
            teste eine schicht viele neuronen 
            eine schicht wenige neuronen
            mehrere schichten mehrere neuronen / mit dropout dazwischen
            viele schichten wenige neuronen /mit dropout dazwischen

            auf Grund des zeitlichen Faktors fallen Deep und multibig weg
            Also zu testen:
            Single Small
            Single 
            Multi Small
            Multi 

        \paragraph{Threadinfo}
            Hypothese:
            Threadinfos bringen was

            Einbinden von thread information auf verschiedenen wegen:
            Thread aware ngrams (tan)
            Thread aware ngrams for w2v (tanw2v)
            Thread change flag (tcf)

            varianten:
            tan
            tanw2v
            tcf
            tan tcf
            tan tanw2v
            tcf tanw2v
            tan tanw2v tcf

            ---> welcher dieser varianten am besten?

        \paragraph{Parameter}
            args
            time

            LSTM ohne Threadinfos mit OHE
            LSTM mit W2V ohne Threadinfos (ngram)
            LSTM mit W2V mit Threadinfos (ngram)
            LSTM mit W2V threadaware mit Threadinfos (ngram)
            LSTM mit W2V threadaware mit Threadinfos (ngram) und threadchangeflag
            LSTM mit W2Vthreadaware mit Threadinfos (ngram) und threadchangeflag, spezialtraining
            --> LSTM final

            Manche angriffe verändern Sequenz von syscalls nicht
            Hypothese:
            verwende Parameter um erg zu verb

            LSTM final + strlen
            LSTM final + time delta
            LSTM final + strlen + time delta
\section{LSTM Ansatz}


\section{Extra Parameter}
    \subsection{Timing}\label{sec:Ergebnis_timing}

