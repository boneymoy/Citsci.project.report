\chapter{Ergebnisse}\label{ch:erg}

In diesem Kapitel sollen die Ergebnisse der Experimente dargestellt und ausgewertet werden.
\autoref{sec:erg_LSTM} untersucht Konfigurationen des \ac{LSTM} ohne Verwendung von Extraparametern.
Dabei werden diese speziell nur entweder nach \ac{DR} oder Anzahl der \acp{FP} eingestuft.
In \autoref{sec:erg_LSTM_extra} werden auch Ergebnisse mit Extraparametern hinzugezogen.
Dabei soll zusätzlich eine beste Konfiguration gefunden werden welche nicht entweder die \ac{DR} oder die \acp{FP} isoliert betrachtet.
Zusätzlich werden in \autoref{sec:erg_vgl} Ergebnisse weiterer Arbeiten mit dieser Arbeit verglichen.
Dabei gibt es wie in \autoref{ch:verwandte_arbeiten} beschrieben nur eine Arbeit die ebenfalls ein \ac{LSTM} auf demselben Datensatz verwenden.


\section{LSTM Ansatz}\label{sec:erg_LSTM}
    Im ersten Schritt der Auswertung wird die \ac{DR} isoliert für Konfigurationen betrachtet welche keine weiteren Extraparameter der System Calls enthalten.
    Alle Konfigurationen verwenden dabei \textit{Thread Aware} n-gramme.
    In \autoref{tab:LSTM_erg} werden die zehn besten Ergebnisse dargestellt.
    Die Durchschnittliche \ac{DR} von $0.69$ des besten Ergebnisses zeigt, dass es generell möglich ist mit \acp{LSTM} eine Anomalieerkennung auf Basis von System Calls umzusetzen.
    Eine klarer Zusammenhang zwischen Embedding Größe (embed) oder n-gram Größe (n-gram) und der \ac{DR} ist dabei nicht direkt zu erkennen.
    So hat die Konfiguration in Zeile $2$ mit einer n-gram Größe von $10$ und einer Embedding Größe von $4$ das \ac{LSTM} $10 \cdot 4=40$ Eingangsneuronen.
    Hingegen hat die Konfiguration aus Zeile $4$ nur $2 \cdot 4 = 8$ Eingangsneuronen.
    Beide Konfigurationen erzielen dabei sehr ähnliche Ergebnisse mit einer durchschnittlichen \ac{DR} von $0.67$ und $15.63$ \acp{FP} beziehungsweise $0.64$ und $13.50$.


    \begin{table}[ht]
        \centering
        \begin{tabular}{cccccc}
            \hline
            \rowcolor{GruvGray!36}
            \multicolumn{6}{c}{Ergebnisse für \ac{LSTM} ohne Extraparameter}\\
            \toprule
            n-gram & embed & \overline{\ac{FP}} & \ac{FP}\_sum & \overline{\ac{DR}} & count\\
            \midrule
            \rowcolor{GruvGray!16}
            $6$ & 	$6$ & 	$22.22$ &  	$209$ & 	$0.69$ &  	$9$ \\
            $10$ & 	$4$ & 	$15.63$ &  	$125$ & 	$0.67$ &  	$8$ \\
            \rowcolor{GruvGray!16}
            $2$ & 	$10$ & 	$6.88$ &  	$55$ & 	    $0.64$ &  	$8$ \\
            $2$ & 	$4$ & 	$13.50$ &  	$108$ & 	$0.64$ &  	$8$ \\
            \rowcolor{GruvGray!16}
            $2$ & 	$6$ & 	$20.60$ &  	$206$ & 	$0.64$ &  	$10$ \\
            $10$ & 	$8$ & 	$19.13$ &  	$153$ & 	$0.62$ &  	$8$ \\
            \rowcolor{GruvGray!16}
            $10$ & 	$6$ & 	$6.44$ &  	$58$ & 	    $0.56$ &  	$9$ \\
            $6$ & 	$4$ & 	$10.13$ &  	$81$ & 	    $0.53$ &  	$8$ \\
            \rowcolor{GruvGray!16}
            $10$ & 	$10$ & 	$5.38$ &  	$43$ & 	    $0.51$ &  	$8$ \\
            $6$ & 	$10$ & 	$7.26$ &  	$58$ & 	    $0.51$ &  	$8$ \\
            \rowcolor{GruvGray!16}
            $2$ & 	$8$ & 	$8.40$ &  	$84$ & 	    $0.50$ &  	$10$ \\
            \bottomrule
        \end{tabular}
        \caption{Konfigurationen mit den $10$ höchsten \acp{DR}. 
                 Dabei wurden nur Konfigurationen betrachtet die keine Extraparameter nutzen.
                 Alle n-gramme sind \textit{Thread Aware}.}
        \label{tab:LSTM_erg}
    \end{table}

    Im zweiten Schritt der Auswertung sollen die besten Ergebnisse nur im Bezug auf die Anzahl der Fehlalarme, also \acp{FP} untersucht werden.
    In \autoref{tab:LSTM_erg_FP} werden wieder \textit{Thread Aware} n-gramme genutzt und nur Ergebnisse mit $\ac{DR}>0.5$ einbezogen.

    \begin{table}[ht]
        \centering
        \begin{tabular}{cccccc}
            \hline
            \rowcolor{GruvGray!36}
            \multicolumn{6}{c}{Ergebnisse für \ac{LSTM} ohne Extraparameter}\\
            \toprule
            n-gram & embed & \overline{\ac{FP}} & \ac{FP}\_sum & \overline{\ac{DR}} & count\\
            \midrule
            \rowcolor{GruvGray!16}
            $10$ & 	$10$ & 	$5.38$ &  	$43$ & 	    $0.51$ &  	$8$ \\
            $10$ & 	$6$ & 	$6.44$ &  	$58$ & 	    $0.56$ &  	$9$ \\
            \rowcolor{GruvGray!16}
            $2$ & 	$10$ & 	$6.88$ &  	$55$ & 	    $0.64$ &  	$8$ \\
            $6$ & 	$10$ & 	$7.26$ &  	$58$ & 	    $0.51$ &  	$8$ \\
            \rowcolor{GruvGray!16}
            $2$ & 	$8$ & 	$8.40$ &  	$84$ & 	    $0.50$ &  	$10$ \\
            $6$ & 	$4$ & 	$10.13$ &  	$81$ & 	    $0.53$ &  	$8$ \\
            \rowcolor{GruvGray!16}
            $2$ & 	$4$ & 	$13.50$ &  	$108$ & 	$0.64$ &  	$8$ \\
            $10$ & 	$4$ & 	$15.63$ &  	$125$ & 	$0.67$ &  	$8$ \\
            \rowcolor{GruvGray!16}
            $10$ & 	$8$ & 	$19.13$ &  	$153$ & 	$0.62$ &  	$8$ \\
            $2$ & 	$6$ & 	$20.60$ &  	$206$ & 	$0.64$ &  	$10$ \\
            \rowcolor{GruvGray!16}
            $6$ & 	$6$ & 	$22.22$ &  	$209$ & 	$0.69$ &  	$9$ \\
            \bottomrule
        \end{tabular}
        \caption{Konfigurationen mit den $10$ niedrigsten \acp{FP}. 
                 Dabei wurden nur Konfigurationen betrachtet die keine Extraparameter nutzen.
                 Alle n-gramme sind \textit{Thread Aware}.
                 Es wurden nur Konfigurationen mit $\ac{DR}>0.5$ einbezogen.}
        \label{tab:LSTM_erg_FP}
    \end{table}
    
    Hierbei ergibt sich ein ähnliches Bild wie zuvor, da es nur wenig mehr als zehn Konfigurationen ohne Extraparameter mit einer $\ac{DR}>0.5$ gibt.
    Doch die Tabelle kann genutzt werden um noch einmal auf die Schwierigkeit der Auswahl der besten Konfigurationen hinzuweisen.
    Ist eine Konfiguration mit hoher \ac{DR} oder eine Konfiguration mit wenigen \acp{FP} zu bevorzugen?
    Oder konkret wie in diesem Fall, ist eine $\ac{DR}=0.69$ bei $22$ Fehlern in grob acht Stunden Testaufnahmen der Konfiguration mit einer $\ac{DR}=0.51$ bei ca. $5$ Fehlern zu bevorzugen?
    Der Versuch eine beste Konfiguration auszuwählen soll nach Betrachtung der Konfigurationen mit Extraparametern unternommen werden.


\section{Extra Parameter}\label{sec:erg_LSTM_extra}
    In \autoref{tab:LSTM_par_erg} werden zusätzlich auch Konfigurationen mit Extraparametern angezeigt. 
    Wieder werden die Ergebnisse nach der besten \ac{DR} eingestuft.
    Als erstes fällt dabei ins Auge, dass alle zehn Ergebnisse besser sind als das beste Ergebnis aus \autoref{tab:LSTM_erg}.
    Dies gibt einen Hinweis darauf, dass die Parameter im Einsatz mit einem \ac{LSTM} sehr nützlich sind.
    Im Vergleich der besten Konfigurationen werden ohne Extraparameter wie beschrieben eine \ac{DR} von $0.69$ bei durchschnittlich $22.22$ Fehlern.
    Die beste Konfiguration mit Extraparametern erreicht eine \ac{DR} von $0.88$ bei durchschnittlich $22.89$ Fehlern.
    Also eine sehr deutliche Verbesserung.
    Auffällig dabei ist, dass wie in \autoref{tab:LSTM_erg} keinerlei zusammenhang zwischen Eingangsneuronen und Ergebnisqualität zu erkennen ist.
    Die Spanne wird sogar noch größer.
    Bei der Konfiguration in Zeile zwei werden $10\cdot8 + 1 + 1=82$ Eingangsneuronen benötigt und in der sechsten Zeile  lediglich $2\cdot4 + 1 = 9$.
    \begin{table}[ht]
        \centering
        \begin{tabular}{ccccccccc}
            \hline
            \rowcolor{GruvGray!36}
            \multicolumn{9}{c}{Ergebnisse für \ac{LSTM} mit Extraparameter}\\
            \toprule
            n-gram & embed & \textit{rv} & \ac{TCF} & \textit{time} & \overline{\ac{FP}} & \ac{FP}\_sum & \overline{\ac{DR}} & count\\
            \midrule
            \rowcolor{GruvGray!16}
            $6$ & 	$8$ & 	$1$ & 	$1$ & 	$1$ & 	$22.89$ &  	$206$ & 	$0.88$ &  	$9$ \\
            $10$ & 	$8$ & 	$0$ & 	$1$ & 	$1$ & 	$18.56$ &  	$167$ & 	$0.76$ &  	$9$ \\
            \rowcolor{GruvGray!16}
            $10$ & 	$4$ & 	$1$ & 	$1$ & 	$1$ & 	$22.33$ &  	$201$ & 	$0.74$ &  	$9$ \\
            $10$ & 	$4$ & 	$0$ & 	$1$ & 	$0$ & 	$5.03$ &   	$41$ & 	    $0.71$ &  	$8$ \\
            \rowcolor{GruvGray!16}
            $6$ & 	$8$ & 	$1$ & 	$1$ & 	$0$ & 	$9.00$ &   	$81$ & 	    $0.70$ &  	$9$ \\
            $2$ & 	$4$ & 	$1$ & 	$0$ & 	$0$ & 	$12.50$ &  	$100$ & 	$0.70$ &  	$8$ \\
            \rowcolor{GruvGray!16}
            $10$ & 	$4$ & 	$1$ & 	$1$ & 	$0$ & 	$12.63$ &  	$101$ & 	$0.70$ &  	$8$ \\
            $10$ & 	$4$ & 	$1$ & 	$0$ & 	$0$ & 	$14.38$ &  	$115$ & 	$0.70$ &  	$8$ \\
            \rowcolor{GruvGray!16}
            $2$ & 	$8$ & 	$1$ & 	$1$ & 	$1$ & 	$14.00$ &  	$140$ & 	$0.70$ &  	$10$ \\
            $6$ & 	$4$ & 	$1$ & 	$1$ & 	$1$ & 	$10.89$ &  	$98$ & 	    $0.69$ &  	$9$ \\
            \rowcolor{GruvGray!16}
            $2$ & 	$6$ & 	$1$ & 	$0$ & 	$1$ & 	$15.70$ &  	$157$ & 	$0.69$ &  	$10$ \\
            \bottomrule
        \end{tabular}
        \caption{Konfigurationen mit den $10$ höchsten \acp{DR}. 
                 Es wurden Konfigurationen mit und ohne Extraparameter betrachtet.
                 Alle n-gramme sind \textit{Thread Aware}.
                 Bei allen Konfigurationen kommen die Extraparameter zum Einsatz.}
        \label{tab:LSTM_par_erg}
    \end{table}

    Beste Ergebnisse nach \ac{FP}-Rate

    \begin{table}[ht]
        \centering
        \begin{tabular}{ccccccccc}
            \hline
            \rowcolor{GruvGray!36}
            \multicolumn{9}{c}{Ergebnisse für \ac{LSTM} mit Extraparameter}\\
            \toprule
            n-gram & embed & \textit{rv} & \ac{TCF} & \textit{time} & \overline{\ac{FP}} & \ac{FP}\_sum & \overline{\ac{DR}} & count\\
            \midrule
            \rowcolor{GruvGray!16}
            $10$ & 	$4$ & 	$0$ & 	$1$ & 	$0$ & 	$5.03$ & 	$41$ & 	$0.71$ & 	$8$ \\
            $6$ & 	$8$ & 	$0$ & 	$0$ & 	$1$ & 	$4.66$ & 	$42$ & 	$0.67$ & 	$9$ \\
            \rowcolor{GruvGray!16}
            $6$ & 	$4$ & 	$0$ & 	$0$ & 	$1$ & 	$5.38$ & 	$43$ & 	$0.64$ & 	$8$ \\
            $2$ & 	$4$ & 	$1$ & 	$1$ & 	$0$ & 	$5.38$ & 	$43$ & 	$0.51$ & 	$8$ \\
            \rowcolor{GruvGray!16}
            $10$ & 	$10$ & 	$0$ & 	$0$ & 	$0$ & 	$5.38$ & 	$43$ & 	$0.51$ & 	$8$ \\
            $10$ & 	$6$ & 	$0$ & 	$1$ & 	$0$ & 	$5.66$ & 	$51$ & 	$0.60$ & 	$9$ \\
            \rowcolor{GruvGray!16}
            $2$ & 	$10$ & 	$0$ & 	$0$ & 	$0$ & 	$6.88$ & 	$55$ & 	$0.64$ & 	$8$ \\
            $6$ & 	$6$ & 	$1$ & 	$1$ & 	$1$ & 	$6.33$ & 	$57$ & 	$0.66$ & 	$9$ \\
            \rowcolor{GruvGray!16}
            $10$ & 	$6$ & 	$1$ & 	$0$ & 	$0$ & 	$6.33$ & 	$57$ & 	$0.55$ & 	$9$ \\
            $10$ & 	$6$ & 	$0$ & 	$0$ & 	$0$ & 	$6.44$ & 	$58$ & 	$0.56$ & 	$9$ \\
            \rowcolor{GruvGray!16}
            $6$ & 	$10$ & 	$0$ & 	$0$ & 	$0$ & 	$7.26$ & 	$58$ & 	$0.51$ & 	$8$ \\
            \bottomrule
        \end{tabular}
        \caption{Konfigurationen mit den $10$ niedrigsten \ac{FP}-Raten. 
                 Es wurden Konfigurationen mit und ohne Extraparameter betrachtet.  Alle n-gramme sind \textit{Thread Aware}.
                 Auch hier kommen bei den meisten Konfigurationen die Extraparameter zum Einsatz.}
        \label{tab:LSTM_FP}
    \end{table}

    Vergleich der zwei besten Konfigurationen (beste \ac{DR} vs beste \ac{FP}-Rate)

    \begin{table}[ht]
        \centering
        \begin{tabular}{cccccc}
            \hline
            \rowcolor{GruvGray!36}
            \multicolumn{6}{c}{Vergleich Konfiguration mit höchster \ac{DR} vs niedrigste \ac{FP}-Rate}\\
            \hline
            Szenario & \ac{FP}-Rate & \ac{DR} & vs & \ac{FP}-Rate & \ac{DR}\\
            \toprule
            \rowcolor{GruvGray!16}
            Bruteforce CWE-307   & $29$ & $0.938776$ & x & $12$ & $0.642857$ \\
            CVE-2012-2122 	      & $18$ & $0.987097$ & x & $14$ & $0.051613$ \\
            \rowcolor{GruvGray!16}
            CVE-2014-0160 	      & $13$ & $0.030000$ & x & $4$  & $0.010000$ \\
            CVE-2017-7529 	      & $0$  & $0.988506$ & x & $1$  & $0.988506$ \\
            \rowcolor{GruvGray!16} CVE-2018-3760 	      & $4$  & $1.000000$ & x & $1$  & $1.000000$ \\
            CVE-2019-5418 	      & $7$  & $1.000000$ & x & $1$  & $0.951456$ \\
            \rowcolor{GruvGray!16}
            PHP CWE-434 	      & $24$ & $1.000000$ & x & $3$  & $1.000000$\\
            SQL Injection CWE-89 &	$82$ & $1.000000$ & x & $5$  & $1.000000$\\
            \rowcolor{GruvGray!16}
            EPS CWE-434 	      & $29$ & $1.000000$ & x & xxx  & xxx \\
            ZIP Slip              & xxx  & xxx        & x & xxx  & xxx \\
            \bottomrule
        \end{tabular}
        \caption{Vergleich bester Ergebnisse.
                Höchste \ac{DR} links mit folgenden Parametern: $n=6, e=8, rv=1, \ac{TCF}=1, time=1$.
                Niedrigste \ac{FP}-Rate rechts mit folgenden Parametern: $n=10, e=4, rv=0, \ac{TCF}=1, time=0$}
        \label{tab:LSTM_vs}
    \end{table}

    Vergleich mit anderen LSTM Ansätzen:

    Vergleich mit \ac{STIDE} von Grimmer et al.~\cite{IDSTHREADGRIMMER2021}.

    Dabei wird im \ac{STIDE} ein \ac{SW} verwendet, dabei wird der Anomaliescore über die Größe de \ac{SW} gemittelt.
    \begin{table}[ht]
        \centering
        \begin{tabular}{lcccccccc}
            \hline
            \rowcolor{GruvGray!36}
            \multicolumn{9}{c}{Ergebnisse für \ac{LSTM} mit Extraparameter}\\
            \toprule
            Algorithmus & n-gram & embed & \ac{SW} & \textit{rv} & \ac{TCF} & \textit{time} & \overline{\ac{FP}} & \overline{\ac{DR}} \\
            \midrule
            \ac{STIDE} & $5$ & int & $1000$ & $0$ & $0$ & $0$ & $61.5$ & $0.986$ \\
            \rowcolor{GruvGray!16}
            best \ac{DR} LSTM & $6$ & 	$8$ & $1$ & 	$1$ & 	$1$ & 	$1$ & 	$22.89$& 	$0.88$ \\
            \rowcolor{GruvGray!16}
            best \ac{FP} LSTM & $10$ & 	$4$ & $1$ &	$0$ & 	$1$ & 	$0$ & 	$5.03$ & 	$0.71$ \\
            \bottomrule
        \end{tabular}
        \caption{}
        \label{tab:LSTM_stide_erg}
    \end{table}

\section{Vergleich anderer Arbeiten}\label{sec:erg_vgl}
Diese arbeitet allerdings mit einer anderen Verteilung der Test- und Trainigsdaten sowie anderen Metriken.
\iffalse
    Durchschnittliche Veränderung bei Hinzunahme time delta
    Durchschnittliche Veränderung bei Hinzunahme \ac{TCF}
    \begin{table}[ht]
        \centering
        \begin{tabular}{c|c|c|c}
            \hline
            \rowcolor{GruvGray!36}
            \multicolumn{4}{c}{Vergleich Nutzung von \textit{time}}\\
            \hline
            Szenario & \textit{time} & $\overline{FP}$ & \overline{\ac{DR}}\\
            \hline
            \hline
            \rowcolor{GruvGray!16}
            Bruteforce CWE-307 &  	$1$ & 	$29.28$ &  	$0.324546$ \\
            \rowcolor{GruvGray!16}
            Bruteforce CWE-307 & 	$0$ & 	$21.78$ & 	$0.251867$ \\
            CVE-2012-2122 & 	        $1$ & 	$10.19$ & 	$0.039606$ \\
            CVE-2012-2122 &      	$0$ & 	$12.29$ & 	$0.018725$ \\
            \rowcolor{GruvGray!16}
            CVE-2014-0160 & 	        $1$ & 	$6.48$ &  	$0.011111$ \\
            \rowcolor{GruvGray!16}
            CVE-2014-0160 & 	        $0$ & 	$4.71$ &  	$0.005854$ \\
            CVE-2017-7529 &       	$1$ & 	$1.50$ &  	$0.906290$ \\
            CVE-2017-7529 & 	        $0$ & 	$0.83$ &  	$0.868517$ \\
            \rowcolor{GruvGray!16}
            CVE-2018-3760 & 	        $1$ & 	$10.14$ & 	$1.000000$ \\
            \rowcolor{GruvGray!16}
            CVE-2018-3760 &       	$0$ & 	$10.04$ & 	$1.000000$ \\
            CVE-2019-5418 &       	$1$ & 	$7.89$ &  	$0.584184$ \\
            CVE-2019-5418 &       	$0$ & 	$7.75$ &  	$0.575255$ \\
            \rowcolor{GruvGray!16}
            EPS CWE-434 &        	$0$ & 	$14.92$ & 	$1.000000$ \\
            \rowcolor{GruvGray!16}
            EPS CWE-434 & 	        $1$ & 	$20.96$ & 	$0.965517$ \\
            PHP CWE-434 &         	$0$ & 	$10.81$ & 	$0.935591$ \\
            PHP CWE-434 & 	        $1$ & 	$14.91$ & 	$0.887810$ \\
            \rowcolor{GruvGray!16}
            SQL Injection CWE-89 & 	$1$ & 	$21.39$ & 	$0.990278$ \\
            \rowcolor{GruvGray!16}
            SQL Injection CWE-89 & 	$0$ & 	$24.90$ & 	$0.974000$ \\
            ZipSlip & 	            $1$ & 	$8.26$ &  	$0.187500$ \\
            ZipSlip & 	            $0$ & 	$21.01$ & 	$0.176020$ \\
        \end{tabular}
        \caption{}
        \label{tab:LSTM_time_erg}
    \end{table}


    \begin{table}[ht]
        \centering
        \begin{tabular}{c|c|c|c}
            \hline
            \rowcolor{GruvGray!36}
            \multicolumn{4}{c}{Vergleich Nutzung von \textit{rv}}\\
            \hline
            Szenario & \textit{rv} & $\overline{FP}$ & \overline{\ac{DR}}\\
            \hline
            \hline
            \rowcolor{GruvGray!16}
            Bruteforce CWE-307 & 	$1$ & 	$24.13$ & 	$0.311094$ \\
            \rowcolor{GruvGray!16}
            Bruteforce CWE-307 & 	$0$ & 	$23.32$ & 	$0 	0.259936$ \\
            CVE-2012-2122 & 	$1$ & 	$13.24$ & 	$0.039728$ \\
            CVE-2012-2122 & 	$0$ & 	$8.51$ & 	$0.017535$ \\
            \rowcolor{GruvGray!16}
            CVE-2014-0160 & 	$1$ & 	$7.34$ & 	$0.011842$ \\
            \rowcolor{GruvGray!16}
            CVE-2014-0160 & 	$0$ & 	$3.49$ & 	$0.004872$ \\
            CVE-2017-7529 & 	$1$ & 	$0.92$ & 	$0.910617$ \\
            CVE-2017-7529 & 	$0$ & 	$1.36$ & 	$0.862364$ \\
            \rowcolor{GruvGray!16}
            CVE-2018-3760 & 	$1$ & 	$9.18$ & 	$1.000000$ \\
            \rowcolor{GruvGray!16}
            CVE-2018-3760 & 	$0$ & 	$9.92$ & 	$1.000000$ \\
            CVE-2019-5418 & 	$1$ & 	$7.47$ & 	$0.684211$ \\
            CVE-2019-5418 & 	$0$ & 	$7.89$ & 	$0.474758$ \\
            \rowcolor{GruvGray!16}
            EPS CWE-434 & 	    $0$ & 	$18.46$ & 	$1.000000$ \\
            \rowcolor{GruvGray!16}
            EPS CWE-434 & 	    $1$ & 	$17.56$ & 	$0.962963$ \\
            PHP CWE-434 & 	    $0$ & 	$11.87$ & 	$0.919343$ \\
            PHP CWE-434 &   	$1$ & 	$12.13$ & 	$0.907001$ \\
            \rowcolor{GruvGray!16}
            SQL Injection CWE-89 & 	$1$ & 	$22.95$ & 	$0.984211$ \\
            \rowcolor{GruvGray!16}
            SQL Injection CWE-89 & 	$0$ & 	$22.11$ & 	$0 	0.979211$ \\
            ZipSlip & 	$0$ & 	$18.25$ & 	$0.196429$ \\
            ZipSlip & 	$1$ & 	$10.50$ & 	$0.167092$ \\
        \end{tabular}
        \caption{}
        \label{tab:LSTM_rv_erg}
    \end{table}


    \begin{table}[ht]
        \centering
        \begin{tabular}{c|c|c|c}
            \hline
            \rowcolor{GruvGray!36}
            \multicolumn{4}{c}{Vergleich Nutzung von \ac{TCF}}\\
            \hline
            Szenario & \ac{TCF} & $\overline{FP}$ & \overline{\ac{DR}}\\
            \hline
            \hline
            \rowcolor{GruvGray!16}
            Bruteforce CWE-307 & 	$1$ & 	$29.67$ & 	$0.402494$ \\
            \rowcolor{GruvGray!16}
            Bruteforce CWE-307 & 	$0$ & 	$21.44$ & 	$0.183425$ \\
            CVE-2012-2122 & 	        $1$ & 	$4.91$ &  	$0.042652$ \\
            CVE-2012-2122 & 	        $0$ & 	$16.92$ & 	$0.016050$ \\
            \rowcolor{GruvGray!16}
            CVE-2014-0160 & 	        $1$ & 	$4.67$ &  	$0.008333$ \\
            \rowcolor{GruvGray!16}
            CVE-2014-0160 & 	        $0$ & 	$6.29$ &  	$0.008293$ \\
            CVE-2017-7529 & 	        $1$ & 	$1.50$ &  	$0.933589$ \\
            CVE-2017-7529 &      	$0$ & 	$0.83$ &  	$0.844547$ \\
            \rowcolor{GruvGray!16}
            CVE-2018-3760 &       	$1$ & 	$8.22$ &  	$1.000000$ \\
            \rowcolor{GruvGray!16}
            CVE-2018-3760 &       	$0$ & 	$11.73$ & 	$1.000000$ \\
            CVE-2019-5418 &       	$0$ & 	$9.73$ &  	$0.707317$ \\
            CVE-2019-5418 &       	$1$ & 	$5.57$ &  	$0.429738$ \\
            \rowcolor{GruvGray!16}
            EPS CWE-434 & 	        $0$ & 	$17.54$ & 	$1.000000$ \\
            \rowcolor{GruvGray!16}
            EPS CWE-434 &         	$1$ & 	$18.79$ & 	$0.965517$ \\
            PHP CWE-434 &         	$1$ & 	$13.72$ & 	$0.935545$ \\
            PHP CWE-434 &        	$0$ & 	$11.86$ & 	$0.893677$ \\
            \rowcolor{GruvGray!16}
            SQL Injection CWE-89 & 	$1$ & 	$32.06$ & 	$1.000000$ \\
            \rowcolor{GruvGray!16}
            SQL Injection CWE-89 & 	$0$ & 	$15.30$ & 	$0.965250$ \\
            ZipSlip &               $0$ & 	$17.76$ &   $0.196429$ \\
            ZipSlip & 	            $1$ & 	$11.51$ & 	$0.167092$ \\
        \end{tabular}
        \caption{}
        \label{tab:LSTM_tcf_erg}
    \end{table}
    \fi


\iffalse
    \section{Optimale Parameter}

            \paragraph{Architektur}
                versch architekturen:
                Single Small 50 neuronen eine schicht
                Single Big 250 neuronen eine schicht
                multi small 20 neuronen 3 schichten
                multi big 50 neuronrn 3 schichten
                deep erste 50 sonst 20 6 schichten

                singlesmall 43\% von Deep
                insgesamt am schnellsten single small
                wie zu erwarten,  deep am langsamsten

            \paragraph{Hyperparameter}<++>
                aktivierungs funktion
                -> dense layer with softmax or tanh
                batch size
                learning rate
                optimizer

            \paragraph{Ngram Größe}
                ngram größer -> langsamer

            \paragraph{Embedding}

                overhead berechnung embedding, muss allerdings nur einmal berechnet werden
                zu erkennen w2v mit embedding size = 2  und window = 4 wesentlich schneller
                embedding größer -> langsamer

                vergleich ngram
                im schnitt mit ngram gr 2 84\% von ngr 3 und 

                w2v bringt entscheidenden Vorteil gegenüber ohe:
                Jeweils vergleich der selben parameter außer w2v vs ohe:
                Single small w2v nur 30\% der zeit gegenüber single small ohe
                bei mulit w2v sogar nur 13\%
                im mittel über alle architekturen 21.5\% der Zeit von ohe bei verwendung w2v

            \paragraph{Architektur}
                teste eine schicht viele neuronen 
                eine schicht wenige neuronen
                mehrere schichten mehrere neuronen / mit dropout dazwischen
                viele schichten wenige neuronen /mit dropout dazwischen

                auf Grund des zeitlichen Faktors fallen Deep und multibig weg
                Also zu testen:
                Single Small
                Single 
                Multi Small
                Multi 

            \paragraph{Threadinfo}
                Hypothese:
                Threadinfos bringen was

                Einbinden von thread information auf verschiedenen wegen:
                Thread aware ngrams (tan)
                Thread aware ngrams for w2v (tanw2v)
                Thread change flag (tcf)

                varianten:
                tan
                tanw2v
                tcf
                tan tcf
                tan tanw2v
                tcf tanw2v
                tan tanw2v tcf

                ---> welcher dieser varianten am besten?

            \paragraph{Parameter}
                args
                time

                LSTM ohne Threadinfos mit OHE
                LSTM mit W2V ohne Threadinfos (ngram)
                LSTM mit W2V mit Threadinfos (ngram)
                LSTM mit W2V threadaware mit Threadinfos (ngram)
                LSTM mit W2V threadaware mit Threadinfos (ngram) und threadchangeflag
                LSTM mit W2Vthreadaware mit Threadinfos (ngram) und threadchangeflag, spezialtraining
                --> LSTM final

                Manche angriffe verändern Sequenz von syscalls nicht
                Hypothese:
                verwende Parameter um erg zu verb

                LSTM final + strlen
                LSTM final + time delta
                LSTM final + strlen + time delta
    \section{LSTM Ansatz}
\fi

    %\subsection{Timing}\label{sec:Ergebnis_timing}
    %\subsection{Return Value}\label{sec:Ergebnis_return}

