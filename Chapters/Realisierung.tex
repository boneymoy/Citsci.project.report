
%*****************************************
\chapter{Realisierung}\label{ch:realisierung}
%*****************************************
    Gegeben 10 szenarien mit ca. 1000 files durschnittlich 45sec
    in runs.csv genauere beschreibung files mit label und zeitangabe falls exploit
    falls kein exploit dann exploit start time -1
    keine dauer des exploits also ende nicht bekannt 
    nicht systemcall genau start des angriffs angegeben 
    führe puffer ein, da angegebener Zeitpunkt ungenau, sodass auch wirklich jeder angriff nach exploit start time
    alles nach dem angrffszeitpunkt muss als anomalie gewertet werden, auch wenn angriff evtl noch nicht gestartet hat oder schon vorbei.
    Ungenauigkeiten auf die in der Auswertung der Daten noch einmal genauer eingegangen wird
    filtern von switch statements in Datensatz weil keine system calls
    nur öffnende syscalls keine schließenden

    Eingehen auf verwendete Tools 

\section{Verwendete Tools}
    Tensorflow Keras
    Rechencluster clara
    sysdig

\section{Vorverarbeitung}

    Vorverarbeitung besteht aus vier Teilbereichen 
    wie wird ein System Call dargestellt nur namen für sequenz
    wie wird stream von daten in diesem Fall system calls abgebildet
    wie werden die Parameter wie return value oder argumente präpariert?
    wie können extrainformationen wie z.B thread awareness dargestellt werden
    Diese werden in den folgenden Abschnitten beschrieben

    \subsection{Wie wird ein System Call dargestellt?}

        Neuronale netze benötigen numerische werte deswegen umwandlung von stringnamen 
        sys to int unbrauchbar für netz auf Grund aktivierungsfunktion --> 2>1 3>1 mittelwert 2
        Darstellung von kategorischen Daten für neuronale Netze bekanntes Problem,
        Auch in der Spracherkennung, hier ist allerdings auch ein ohe vorstellbar, da begrenzte Anzahl an versch
        System calls.
        %TODO Abschnitt ohe
        Eine weitere Möglichkeit System Calls darzustellen wird ebenfalls in der Spracherkennung verwendent.
        Durch \textit{Word embeddings} kann zusätzlich zu der Kategorie auch versucht noch der Konext des Wortes im in der Repräsentation einzubinden.
        %TODO Abschnitt w2v

        \paragraph{Embedding}
            ohe und w2v
            word embedding parameterwahl wichtig $sqrt(distinct)$
            Threadid kodieren: 
            \begin{itemize}
                \item use entity embedding for ThreadID~\cite{GUO2016} 
                \item relationship between threads and reduce size (possible 1000 different threads)
                \item choose size of embedding -thumbrule $sqrt(unique value)$
            \end{itemize}
            zeit kodieren
            \begin{itemize}
                \item use time delta of two different syscalls as new input
            \end{itemize}
            parameterlänge kodieren
            \begin{itemize}
                \item syscall to int: Wandle Syscall name in Integer um

                 ohe of sysint: use ohe for every syscall 
                 n * (distinct calls + 1) eingabeneuronen 

                 w2v von syscall
                 weniger neuronen und nähe von syscall!!!
                \item ngram bilden: Bilde entsprechend angegebenes n ngramm
            \end{itemize}
            overhead berechnung embedding, muss allerdings nur einmal berechnet werden
            zu erkennen w2v mit embedding size = 2  und window = 4 wesentlich schneller
            embedding größer -> langsamer
            vergleich ngram
            im schnitt mit ngram gr 2 84\% von ngr 3 und 
            w2v bringt entscheidenden Vorteil gegenüber ohe:
            Jeweils vergleich der selben parameter außer w2v vs ohe:
            Single small w2v nur 30\% der zeit gegenüber single small ohe
            bei mulit w2v sogar nur 13\%
            im mittel über alle architekturen 21.5\% der Zeit von ohe bei verwendung w2v

        Nachdem nun eine Darstellung eines System Calls besprochen wurde muss auch noch auf die Streamverarbeitung eingegangen werden.

    \subsection{Wie wird ein Datenstream dargestellt}
        ngrams
        streamingwindow
    %%%%NOTES
    % iterate syscalls 
    % Filter switch
    % Thread awareness ngrams
    % create embedding of syscall
    % append extra parameters

    \subsection{Wie können weitere System Call Informationen dargestellt werden?}
        \paragraph{System Call Argumente}  
            most frequent calls
            Sql: vfrom, fcntl, lstat, nmap, poll
            php: vfrom, fcntl, lstat, nmap, poll
            bru: writev,read,  close, nmap, poll
            eps: open,  read,  fstat, mmap, brk
            zip: futex, write, getpid,protect,open
            2019: statat,write, futex, stat, getpid
            2014: writev,read,  nmap,  close,poll
            2017: lwait,close,ollctl,write,writev
            2018: statat,write, futex, getpid, stat


            unique:
                vfrom, fcntl, lstat, nmap, poll, writev, read, close, open, fstat, mmap, brk
                futex, write, getpid, protect, statat, stat, lwait, ollctl

        %\begin{table}[ht]
            %\small
            %\label{tab:syscall}
            %\centering
            %\begin{tabular}{c||p{6cm}|p{3cm}|p{3cm}}
                %\hline
                %\rowcolor{Gray!36}
                %\multicolumn{3}{c}{System Calls}\\
                %\hline
                %Name & Beschreibung & Interessante Argumente & return\\
                %\hline
                %\hline
                %\rowcolor{Gray!16}
                %fcntl & manipulate current fd & None & None
                %\hline
                %lstat & file status & path & 
                %\rowcolor{Gray!16}
                %mmap & map or unmap files or devices into memory & addr, length, fd,... & pointer to maped area
                %poll & wait for event on fd & fds (fd, events, revents), timeout
                %\rowcolor{Gray!16}
                %open& Öffnet die in \textit{path} spezifizierte File und gibt einen \textit{file descriptor} zurück.& \textit{path}, \textit{flags}, \textit{mode} \\
                %write& Schreibt bis zu \textit{count} Bytes aus dem Buffer (ab Stelle \textit{buf}) in die File, welche über den \textit{file descriptor}$fd$ definiert wird. & $fd$, $*buf$, \textit{count} \\
                %\hline
            %\end{tabular}
            %\caption{Beschreibung ausgewählter System Calls}
        %\end{table}
        interesting arguments:
        
        \paragraph{Thread Awareness}
            \textit{Thread aware ngrams}
                syscalls welche nicht in trainingsdaten bekommen eine 0
                ngramme thread aware bilden unbrauchbar bei ngram länge von 1
            \textit{Thread change flag}
                oder thread change flag

        \paragraph{Zeitliche Abstände von System Calls}
            Berechne größten Abstand zwischen zwei aufeinanderfolgende System Calls in den Trainingsdaten.
            Normiere folgende System Call Abstände mit diesem Maximum.

        \paragraph{Return Werte}
            Anzahl unique return Werte, nur numerische Werte betrachtet, 
            res=-11(<f>/path/to/some/file) wird als -11 interpretiert
            Bruteforce: 17192 2851 int, 14333 hex, 8 else
            2012:       1897 193 int, 1698 hex, 6
            2014:       16736 2706, 14022, 8
            2017:       381 26, 354, 1 
            2018:       110 106, 0, 4 
            2019:       125 119 int werte und 6 else
            SQL
            EPS
            PHP
            Zipslip

            Hinweis 3 Kategorien evtl kategorische Einteilung bei geringer Anzahl
            Mögliche Bedeutungen bei else:
                Meist Fehlermeldungen!!!
                stat:
                    file status, 
                    sollte 0 bei success und -1 bei error, gefunden auch -2 
                ioctl:
                    manipulate underlying device, 
                    normalerweise 0 bei success manchmal negative werte
                    -1 bei error
                    manchmal return als ausgabe parameter
                Kategorische Betrachtung sinnvoll
            HEX:\@
                mmap:
                    map or unmap files into memory
                    return pointer to mapped area
                brk:
                    change data segment size
                    return 0 on success -1 on error
                    got hex value

            Systemcalls welche Rückgabewerte haben:
                %Bruteforce: {'select': 11012, 
                             %'poll': 64443,
                             %'close': 80535,
                             %'read': 92587,
                             %'epoll_wait': 13746,
                             %'semop': 27435,
                             %'mmap': 45437,
                             %'fcntl': 35722,
        %'brk': 30372, 'writev': 126627, 'stat': 28356, 'fstat': 12950, 'getcwd': 4616, 'lstat': 8060, 'write': 39703, 'chdir': 4616, 'munmap': 90757, 'rename': 2308, 'shutdown': 11127, 'connect': 4106, 'getdents': 40, 'clone': 1409, 'setgid': 705, 'lseek': 1410, 'setuid': 705}
                %2012: {'futex': 149729, 'select': 7846, 'poll': 1262, 'fcntl': 147382, 'setsockopt': 148874, 'mmap': 1723, 'clone': 2523, 'munmap': 464, 'write': 71308, 'read': 209140, 'access': 3337, 'pwrite': 25132, 'stat': 960, 'shutdown': 959, 'close': 959}
                %2014: {'poll': 62797, 'writev': 122406, 'shutdown': 11112, 'select': 11094, 'read': 90368, 'close': 79167, 'epoll_wait': 13629, 'semop': 27166, 'mmap': 44653, 'fcntl': 35692, 'brk': 28193, 'stat': 26827, 'munmap': 90471, 'write': 38219, 'connect': 4231, 'fstat': 12290, 'getcwd': 4324, 'lstat': 7826, 'chdir': 4324, 'rename': 2162, 'clone': 1480, 'setgid': 740, 'lseek': 1480, 'setuid': 740, 'getdents': 34} 
                %2017: {'epoll_wait': 53355, 'recvfrom': 26108, 'fstat': 26385, 'pread': 26109, 'writev': 26110, 'sendfile': 26110, 'write': 26109, 'close': 52823, 'getdents': 552, 'stat': 276, 'brk': 368, 'futex': 92, 'sendmsg': 166, 'recvmsg': 330} 
                %2018: {'select': 56420, 'fcntl': 27396, 'fstat': 82226, 'futex': 262846, 'recvfrom': 27403, 'getsockopt': 27405, 'stat': 252081, 'write': 576100, 'ioctl': 54824, 'lseek': 54824, 'read': 119433, 'close': 82304, 'setsockopt': 54836, 'poll': 4654, 'lstat': 3, 'pread': 3} 
                %2019: {'select': 75772, 'fcntl': 27245, 'fstat': 81543, 'futex': 359158, 'recvfrom': 27145, 'getsockopt': 27129, 'stat': 307293, 'write': 542742, 'ioctl': 54282, 'lseek': 54250, 'read': 136356, 'close': 81599, 'setsockopt': 54269, 'poll': 9050, 'lstat': 245, 'pread': 21, 'unlink': 21, 'rename': 32, 'getdents': 32}
\subsection{Parameterwahl}
        ngram länge
        lstm merkt sich vorherige syscalls aber hinzunahme von syscalls weitere info
        -> finden von sweet spot
        generell großes n viele alarme
        kleines n weniger alarme  vorteil LSTM\@?
        wichtiger Parameter den es zu ermitteln gilt

\section{Algorithmus}\label{sec:Algorithmus}
    LSTM Sprachmodell soll Wahrscheinlichkeit des nächsten System Calls vorhersagen, gegeben eines System Calls oder einer Sequenz von System Calls.
    Gab es in den Trainingsdaten die feste Menge $S = {1,\dots,N}$ an System Calls, so gibt $x=x_1\dots x_l \ (x_i\in S)$ die Sequenz an $l$ System Calls an.
    Jeder dieser System Calls bekommt im ersten Schritt einen Integerwert zwischen 1 und $N$.
    Taucht in den Testdaten nun ein noch nicht bekannter System Call $x_i$ auf, also $x_i \notin S$, so erhält dieser den vorläufigen Wert 0.
    Zu jedem Zeitpunkt wird $x_i$ der Input Layer übergeben.
    Dabei wird ein Embedding aus Abschnitt~\ref{sec:embedding} verwendet. 
    Mit den gegebenen Trainingsdaten kann nun das LSTM mittels des \textit{back-propagation through time} (BPTT) trainiert werden.
    % Im ersten Schritt besteht dieses Embedding aus dem \textit{One hot encoding} (OHE).
    % In weiterer Ausführung werden dann zwei W2V Verfahren verwendet.
    % Wie in Kapitel \ref{Grundlagen:LSTM} bereits beschrieben wird, soll das LSTM mit den kodierten system calls aus dem Trainingsdatensatz trainiert werden.
    An der Ausgangs Layer befindet sich eine Softmax Aktivierungsfunktion.
    Diese wird verwendet um die Ausgabe zu normalisieren und damit die Wahrscheinlichkeitsverteilung des nächsten System Calls zu erhalten.
    Also $P\left(x_i|x_{1:i-1}\right)$ für alle $i$. 
    
    \subsection{Anomalieerkennung}
        Es kann also bei Auftreten des System Calls $x_i$ überprüft werden mit welcher Wahrscheinlichkeit $p$ dieser vorhergesagt wurde.
        Der eigentliche Anomalie-Score wird dann folgenderweise berechnet:
        \begin{equation}
            ascore = 1 - p
        \end{equation}
        Unterschreitet dieser einen Schwellwert so wird dies als eine Anomalie gewertet und ein Alarm angezeigt.
        \subsection{Schwellwertbestimmung}
        Um den zuvor erwähnten Schwellwert automatisch zu bestimmen, wird der Algorithmus auf die Validierungsdaten angewendet. 
        Dabei dient der höchste Wert dieser Daten dann als Schwellwert, da angenommen wird, dass mindestens alle Daten aus den Validierungsdaten harmlos sind und damit unter dem Schwellwert liegen sollten.
        Wichtig ist dabei dafür nicht die Trainingsdaten zu wählen, da eine starke Verzerrung der Schwellwertes durch Overfitting der Daten entstehen könnte. 
        Das würde bedeuten, dass nur sehr geringe Anomaliewerte auftreten und der Schwellwert sehr gering ist und damit die Gefahr für viele Fehlalarme besteht.

        Alternativ betrachte die x wahrscheinlichsten vorhergesagten system calls, falls tatsächlicher system call nicht dabei --> alarm
        x ermitteln, betrachte validierungsdaten und schaue ob schlechtestes x aussehen würde
        tatsächlich oft einmal schlechteste platzierung und automatische erkennung von x schwer.

        \paragraph{Problem des Datensatz}

            In den Testdaten sind \textit{malicious} \marginpar{zu dt.\ schädlich} Files beinhaltet, welche eine Information über den Angriffszeitpunkt liefert. 
            Jedoch gibt es bei einer malicious File im Gegensatz zu den normalen Files  vier mögliche Zuordnungen.
            %TODO Bild einfügen von Quadranten 
            Befindet sich der Anomaliescore unter dem Schwellwert kann von einem \textit{True Negative} eingestuft werden.
            Also es wurde korrekter weise kein Alarm vorhergesagt.
            Befindet sich der Anomaliescore vor dem Angriffszeitpunkt über dem Schwellwert liegt ein \textit{False Positive} vor.
            Es wurde ein Alarm gemeldet an einer Stelle an dem kein Angriff stattfand.
            Nach dem angegebenen Angriffszeitpunkt wird es allerdings schwieriger.
            Denn liegt der Anomaliescore nach dem Angriffszeitpunkt über dem Schwellwert, wird von einem \textit{True Positive}, also einem korrektem Alarm ausgegangen.
            Jedoch könnte da der Angriff schon vorbei sein, oder gar noch nicht gestartet sein.
            Es können nach dem Angriffszeitpunkt auch \textit{False Positive} oder \textit{False Negative} geben, welche allerdings nicht als solche erkannt werden können.
            Wie sich das auf die Auswertung der Ergebnisse auswirkt wird in Kapitel~\ref{sec:metrik} beleuchtet.

\section{Strukturierung der Experimente}
    Um aussagekräftige Experimente zu entwickeln müssen zuerst 
    überlegungen zur praktischen umsetzung gemacht werden
    dabei wird in ersten Tests klar, dass zeit hierbei eine große rolle spielen wird

    erste Tests also ausgelegt um Faktoren zu ermitteln, welche die auswertungen stark verlangsamen
    und diese ausschließen.

    \subsection{Faktor Zeit}

        zeit/dr als groesse und farbe von scatter plot
        batch size test und train x/y achse

        eingrenzen von moeglichen konfigurationen

        Berechnungszeiten aus verschiedenen Perspektiven relevant:
        soll live system werden
        begrenzte rechenleistung und viele Tests zur auswertung von parametern architektur etc
        erster test zur abschätzung diverser zeitl.\ faktoren:

        Faktoren:
        \begin{itemize}
            \item Architektur
            \item Verarbeitung Stream

                 ngram größe
            \item embedding
        \end{itemize}

        ngram größe, architektur und verwendung w2v statt ohe
        Grobe Abschätzung der Zeit, da Berechnungen auf Clustern ausgeführt werden von Auslastung beeinflusst werden.
        Klare Erkenntnisse:

            Single Small 50 neuronen eine schicht:
            Single Big 250 neuronen eine schicht
            multi 50 neuronrn 3 schichten

        erste Abschätzung von Nutzen von Thread 
        einführen von stateful sowie Batch Normalization

    \subsection{Optimale Parameter}

        \paragraph{Architektur}
            versch architekturen:
            Single Small 50 neuronen eine schicht
            Single Big 250 neuronen eine schicht
            multi small 20 neuronen 3 schichten
            multi big 50 neuronrn 3 schichten
            deep erste 50 sonst 20 6 schichten

            singlesmall 43\% von Deep
            insgesamt am schnellsten single small
            wie zu erwarten,  deep am langsamsten

            teste eine schicht viele neuronen 
            eine schicht wenige neuronen
            mehrere schichten mehrere neuronen / mit dropout dazwischen
            viele schichten wenige neuronen /mit dropout dazwischen

            auf Grund des zeitlichen Faktors fallen Deep und multibig weg
            Also zu testen:
            Single Small
            Single 
            Multi Small
            Multi 

        \paragraph{Hyperparameter}
            aktivierungs funktion
            -> dense layer with softmax or tanh
            batch size
            learning rate
            optimizer

        \paragraph{Ngram Größe}
            ngram größer -> langsamer

        \paragraph{Threadinfo}
            Hypothese:
            Threadinfos bringen was

            Einbinden von thread information auf verschiedenen wegen:
            Thread aware ngrams (tan)
            Thread aware ngrams for w2v (tanw2v)
            Thread change flag (tcf)

            varianten:
            tan
            tanw2v
            tcf
            tan tcf
            tan tanw2v
            tcf tanw2v
            tan tanw2v tcf

            ---> welcher dieser varianten am besten?

        \paragraph{Parameter}
            args
            time

            LSTM ohne Threadinfos mit OHE
            LSTM mit W2V ohne Threadinfos (ngram)
            LSTM mit W2V mit Threadinfos (ngram)
            LSTM mit W2V threadaware mit Threadinfos (ngram)
            LSTM mit W2V threadaware mit Threadinfos (ngram) und threadchangeflag
            LSTM mit W2Vthreadaware mit Threadinfos (ngram) und threadchangeflag, spezialtraining
            --> LSTM final

            Manche angriffe verändern Sequenz von syscalls nicht
            Hypothese:
            verwende Parameter um erg zu verb

            LSTM final + strlen
            LSTM final + time delta
            LSTM final + strlen + time delta



\section{Metriken}\label{sec:metrik}

    Auf Grund dessen Metrik False Alarm/ consecutive false alarm und Detection rate falls einmal pro malicious file in quadrant 4 -> HIT
    Wahl von Metriken in NN
    Precision, Recall, f-score, TNR, FNR, FPR

    problematisch:
    nicht auf systemcall genau gelabelt
    recall precision usw nur auf file ebene:
    alarm nach exploitstarttime wird immer als hit gewertet -> aber evtl angriff noch nicht begonnen
    oder angriff bereits vorbei
    ebenso umgekehrt, eig muss jeder nicht alarm nach exploitstart als FN gewertet werden
    weswegen filegenau geschaut wird
    vorteil des Datensatzes gegenüber anderen, immerhin exploitstart time

    alarm in quadrant ---> image




